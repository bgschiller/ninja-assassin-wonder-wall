\documentclass[12pt]{article} 

%math commands and symbols
\usepackage{amsmath}
%\usepackage{amssymb}

%\usepackage{amsthm}
%\newtheorem*{lem}{Lemma}
%\newtheorem{thm}{Theorem}
%\renewcommand{\qedsymbol}{$\blacksquare$}

%%%%%%% allows augmented matrices defined as 
%%% \begin{pmatrix}[cc|c] %or bmatrix, or matrix...
\makeatletter
\renewcommand*\env@matrix[1][*\c@MaxMatrixCols c]{%
  \hskip -\arraycolsep
  \let\@ifnextchar\new@ifnextchar
  \array{#1}}
\makeatother

%used following an integral for proper spacing and font
\newcommand{\ud}{\,\mathrm{d}}

%begins paragraphs with an empty line instead of a tab.
\usepackage[parfill]{parskip}  

%smaller section titles
%\usepackage[compact,small]{titlesec}

%creates smaller margins
%\usepackage{fullpage} 

\usepackage[margin=1in]{geometry}

%in older versions of latex, replace [pdf] with auto-pst-pdf
%\usepackage{auto-pst-pdf}
\usepackage[pdf]{pstricks}
%\usepackage{pst-plot}
%\usepackage{pst-node}
%\usepackage{pst-tree}
%\usepackage{qtree}

%pseudocode typeset as in CLRS Intro to Algorithms
%\usepackage{clrscode3e}

%allows you to put boxes around stuff
%\usepackage{framed}

%allows for comment blocks and verbatim sections
%\usepackage{verbatim} 
%permits use of verbatimtab environment, which preserves tabbing.
%\usepackage{moreverb}

%Include columns
\usepackage{multicol}

%allows for \includegraphics
%\usepackage{graphicx}

%clickable URLs, sections names
%\usepackage[pdfborder={0 0 0}]{hyperref}

\title{Undergraduate Research Proposal}
\author{Brian Schiller}
\date{\today}


\begin{document}
\maketitle

I would like to study a problem that comes from a camp game called `Ninja-Assassin Wonder-Wall'. The game works like this: $n$ players all stand in a circle. Each player chooses two other players  (without indicating their choices), one to be their Ninja-Assassin and the other to be their Wonder-Wall. When the game starts every player tries to keep their Wonder-Wall between themself and their Ninja Assassin.

%solved game
\begin{center}
\begin{multicols}{2}
\begin{pspicture}(0,0)(10,10)
\qdisk(1,1){2pt}
\uput[u](1,1){C}

\qdisk(0,2){2pt}
\uput[u](0,2){E}

\qdisk(2,2){2pt}
\uput[u](2,2){B}

\qdisk(3,3){2pt}
\uput[u](3,3){A}

\qdisk(4,2){2pt}
\uput[u](4,2){D}

\qdisk(6,2){2pt}
\uput[u](6,2){F}
\end{pspicture}

\columnbreak

\begin{tabular}{l | c | c}
 & WW & NA \\
 \hline
 A &  B& C\\
 B &  D& F\\
 C &  B& A\\
 D &  B& E\\
 E &  D& F\\
 F &  D& B
 \end{tabular}
 \end{multicols}
 \small Above is a game in which all players are satisfied with their position.
 \end{center}
 
 
%unsolvable game
\begin{center}
\begin{multicols}{2}
\begin{pspicture}(0,0)(10,10)
\qdisk(1,1){2pt}
\uput[u](1,1){C}

\qdisk(1,2){2pt}
\uput[u](1,2){B}

\qdisk(1,3){2pt}
\uput[u](1,3){A}

\qdisk(0,2){2pt}
\uput[u](0,2){D}

\end{pspicture}

\columnbreak

\begin{tabular}{l | c | c}
 & WW & NA \\
 \hline
 A & B& C\\
 B & D& A\\
 C & B& A\\
 D & B& A
 \end{tabular}
 \end{multicols}
 \small Above is a game where there is no orientation which satisfies all players.
 \end{center}


Because I came up with this problem, and it is not the specialty of any professor, it is important to plan for the event that I become blocked. I can see many different directions to explore with this problem---hopefully enough so that if I'm blocked on one approach I can advance from another angle.

Some directions:
\begin{description}
\item[Counting Problem:] How many $n$ player games are there, where two games are the same if one can be turned into the other by a renaming of the players. For how many of those games is there a solution, an arrangement where all players are satisfied with their position? How does that proportion change with $n$?
\item[Constructive Algorithm:] Given a relationship table, how can we determine if there is a straight-line solution (all players arranged in a straight line with every player satisfied)? Can we give an efficient algorithm to construct such a solution?
\item[Pursuit curves, Diff-Eq:] Create animations of games in two ways:
	\begin{itemize}
	\item ``Continuously'' to create pursuit curves where every player moves toward the ray formed between their Ninja-Assassin and their Wonder-Wall.
	\begin{center}
	\begin{pspicture}(0,0)(10,10)
	\psline{*->}(3,5)(3.4,4.2)
	\uput[ul](3,5){A}
	\uput[dl](3,5){A goes}
	\uput[dl](3.1,4.6){thatta-way}
	
	\psline{*-}(4,3)(1,0)
	\uput[r](4,3){A's WW}
	
	\psline[linestyle=dashed]{*-*}(5,4)(4,3)
	\uput[ur](5,4){A's NA}
	\end{pspicture}
	\end{center}
	\item ``Discretely'' At each state, every player chooses a new lattice point, then all players jump to their chosen point to start the new stage.
	\end{itemize}
\end{description}



\end{document}
