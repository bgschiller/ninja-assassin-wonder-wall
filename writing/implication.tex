\documentclass{article}

\usepackage{clrscode3e}

\usepackage[parfill]{parskip}
\usepackage{amsmath,amsthm,amssymb}
\usepackage{latexsym} %gives nice \leadsto
\newtheorem*{lem}{Lemma}
\newtheorem*{thm}{Theorem}

\DeclareMathOperator{\w}{w}
\DeclareMathOperator{\n}{n}
\DeclareMathOperator{\pairs}{pairs}
\usepackage{moreverb}
\begin{document}

An instance of a Ninja Assassin Wonderwall problem is described by a set $P$ of $n$ players, and two functions $\w : P \to P, \quad \n : P \to P$, that satisfy $\w(p) \neq p$, $\n(p) \neq p$, and $\w(p) \neq \n(p)$ for all $p\in P$. We can construct a directed graph $G=(P,E)$, where we interpret an edge $(a,b) \in E$ to represent the statement `player $a$ is to the left of player $b$ in an ordering of players'. 

This says, transitively, that if there is a path from $a$ to $b$, then $a$ is to the left of $b$ (we will use $a \leadsto b$ to denote both of these). Notice that if these graphs actually represent an ordering of players, they should not contain cycles. For example, when both $(a,b)$ and $(b,a)$ are in $E$, this says $a \leadsto b$ and $b \leadsto a$, which is a contradiction. 


\begin{thm}
A Ninja Assassin Wonderwall problem has a solution iff there is a directed acyclic graph $G=(P,E)$ such that, for all $p\in P$, either  $p \leadsto \w(p) \leadsto \n(p)$ or $\n(p) \leadsto \w(p) \leadsto p$.
\end{thm}
\begin{proof} \mbox{}

\begin{description}
\item[$(\Rightarrow)$] Let $\phi: P \to \mathbb{R}$ be a solution to a Ninja Assassin Wonderwall problem. Construct a graph $G$ with an edge $(p,q)$ whenever $\phi(p)<\phi(q)$. By construction, $G$ is acyclic (in fact it is an ordered graph).

Consider an arbitrary player $p$. Since $\phi$ is a solution, $\phi(\w(p))$ must be between $\phi(p)$ and $\phi(\n(p))$. So either $\phi(p) < \phi(\w(p)) < \phi(\n(p))$ or $\phi(\n(p)) < \phi(\w(p)) < \phi(p)$. Then either $p \leadsto \w(p) \leadsto \n(p)$ or $\n(p) \leadsto \w(p) \leadsto p$.

\item[$(\Leftarrow)$] Let $G$ be such a graph. Let $\{p_i\}$ be a topological ordering of $G$. Define $\phi(p_i) = i$. 

Let $p_i$ be an arbitrary player. By assumption, either $p_i \leadsto \w(p_i) \leadsto \n(p_i)$ or $\n(p_i) \leadsto \w(p_i) \leadsto p_i$. Without loss of generality, $p_i \leadsto \w(p_i) \leadsto \n(p_i)$. Since $\{p_i\}$ is a topological sort on $G$, $p_i$ must come before $\w(p_i)$, which must come before $\n(p_i)$. So $\phi(p_i) < \phi(\w(p_i)) < \phi(\n(p_i))$. Thus, $\phi$ is a solution to a Ninja Assassin Wonderwall problem.
\end{description}
\end{proof}

This proof leads to an algorithm for constructing solutions. Consider the set $D$ of all digraphs $G=(P,E)$, where $E$ is constructed by taking, for all $p \in P$, either  $(p,\w(p)), (\w(p),\n(p))$ or $(\n(p),\w(p)),(\w(p),p)$. There are $2^n$ such graphs, and if a NAWW problem has a solution, one of them must be acyclic.

\begin{codebox}
\Procname{$\proc{FindOrdering}(P,n,w)$}
\li \For $G\in D$
\li     \Do
        \If $G$ is acyclic
\li         \Do
                \Return $\proc{TopologicalSort}(G)$
        \End
    \End
\li \Return None
\end{codebox}
    
There are many small efficiencies we can add to this algorithm. We can avoid enumerating all the elements of $D$ by trying to construct our graph from scratch. We will start with an empty graph $G$. Then we will choose a $p \in P$, add the edges $(p,\w(p))$ and $(\w(p),\n(p))$ and recurse. If this leads to a cyclic graph, we will try the opposite edges and recurse. If \emph{this} leads to a cyclic graph, then neither $p \leadsto \w(p) \leadsto \n(p)$ nor $\n(p) \leadsto \w(p) \leadsto p$ is consitent with the graph, and so there is no solution.

\begin{codebox}
\Procname{$\proc{FindOrdering}(P,n,w)$}
\li $G \gets (\emptyset, \emptyset)$
\li $S \gets P$
\li $G \gets \proc{MakeAssumption}(G,S,n,w)$
\li \If $G$ is acyclic
\li     \Do
        \Return $\proc{TopologicalSort}(G)$
    \End
\li \Return None
\end{codebox}


\begin{codebox}
\Procname{$\proc{MakeAssumption}(G,S,\n,\w)$}
\li \If $S \isequal \emptyset$
\li     \Then
        \Return $G$ \Comment Success!
    \End
\li \If $G$ contains a cycle
\li     \Then
        \Return $G$ \Comment Failure case
    \End
\li Let $p \in S$
\li $F \gets G$
\li $\attrib{F}{E} \gets \attrib{F}{E} \cup \{(p,\w(p)), (\w(p),\n(p))\}$
\li $F \gets \proc{MakeAssumption}(F,S\setminus \{p\}, \n, \w)$
\li \If $F$ is acyclic
\li     \Do
        \Return $F$ \Comment Success!
    \End
\li $F \gets G$ \Comment That didn't work, try the other way.
\li $\attrib{F}{E} \gets \attrib{F}{E} \cup \{(\n(p), \w(p)), (\w(p),p)\}$
\li \Return $\proc{MakeAssumption}(F,S\setminus \{p\}, \n, \w)$ \Comment acyclic or not, this is our best shot.
\end{codebox}

We can make another small improvement by noticing that if $a \leadsto b$ where $a,b \in \{p, \w(p), \n(p)\}$ for some $p$, is already in the graph, then we can add the third element of $\{p, \w(p), \n(p)\}$ immediately. For example if $p \leadsto \n(p)$, then if $G$ is a subgraph of the acyclic graph in the theorem, $p \leadsto \w(p)$ and $\w(p) \leadsto \n(p)$. We will use the same definition of $\proc{FindOrdering}$.

\begin{codebox}
\Procname{$\proc{MakeAssumption}(G,S,\n,\w)$}
\li \If $S = \emptyset$
\li \Do \Return $G$ \Comment Success!
    \End
\li \If $G$ contains a cycle
\li     \Then
        \Return $G$ \Comment Failure case
    \End
\li Let $p \in S$.  			\label{li:choose-p}
\li $F \gets G$
\li $F \gets \proc{AddImpliedEdges}(F, (p,\w(p)), S,\n,\w)$
\li $F \gets \proc{MakeAssumption}(F,S\setminus \{p\}, \n, \w)$
\li \If $F$ is acyclic
\li     \Do
        \Return $F$ \Comment Success!
    \End
\li $F \gets G$ \Comment That didn't work, try the other way.
\li $F \gets \proc{AddImpliedEdges}(F, (\w(p), p), S, \n, \w)$
\li \Return $\proc{MakeAssumption}(F,S\setminus \{p\}, \n, \w)$ \Comment acyclic or not, this is our best shot.
\end{codebox}


\begin{codebox}
\Procname{$\proc{AddImpliedEdges}(F,(a,b),S,\n,\w)$}
\li $Q \gets \emptyset$
\li $\proc{Enqueue}(Q,(a,b))$
\li \While $Q$ is not empty
\li     \Do
        $(r,s) \gets \proc{Dequeue}(Q)$
\li     $\attrib{F}{E} \gets \attrib{F}{E} \cup \{(r,s)\}$
\li     \For $(t,u) \in \proc{ImpliedEdges}((r,s),S,\n,\w)$
\li         \Do
            \If $(t,u) \notin \attrib{F}{E}$
\li             \Do
                $\attrib{F}{E} \gets \attrib{F}{E} \cup \{(t,u)\}$
\li             $\proc{Enqueue}(Q,(t,u))$
            \End
        \End
    \End
\li \Return $F$
\end{codebox}



\begin{codebox}
\Procname{$\proc{ImpliedEdges}((r,s),S,\n,\w)$}
\li $E \gets \emptyset$
\li \For $t \in S$
\li     \Do
        \If $(r,s) \isequal (t, \w(t))$
\li         \Do
            $E \gets E \cup \{(\w(t),\n(t)), (t, \n(t))\}$
\li     \ElseIf $(r,s) \isequal (\w(t), t)$
\li         \Do
            $E \gets E \cup \{(\n(t),\w(t)), (\n(t), t)\}$ 
\li     \ElseIf $(r,s) \isequal (t, \n(t))$
\li         \Do
            $E \gets E \cup \{(t,\w(t)), (\w(t), \n(t))\}$
\li     \ElseIf $(r,s) \isequal (\n(t), t)$
\li         \Do
            $E \gets E \cup \{(\w(t), t), (\n(t), \w(t))\}$
\li     \ElseIf $(r,s) \isequal (\w(t), \n(t))$
\li         \Do
            $E \gets E \cup \{(t,\w(t)), (t, \n(t))\}$
\li     \ElseIf $(r,s) \isequal (\n(t), \w(t))$
\li         \Do
            $E \gets E \cup \{(\w(t), t), (\n(t), t)\}$
        \End
    \End
\li \Return $E$
\end{codebox}

Finally, notice that every \emph{implied edge} we add potentially cuts our running time in half. We no longer need a tree of $2^{\lvert S \rvert}$ recursive calls to determine the direction of that player's constraint. Each edge we add that has an \emph{implied edge} gives us some extra leverage over the problem. Heuristically, it seems like we should exert this leverage as early as possible. We accomplish this by choosing our $p\in S$ on line~\ref{li:choose-p} such that the set $\{p,\w(p), \n(p)\}$ shares two elements with as many $\{t, \w(t), \n(t)\}, t\in S$ as possible.
\begin{comment}
Let $P$ be the set of $n$ players.\\
Let $R$ be a set of $n$ triplets $(x,y,z)$ where $y=\w(x), z=\n(x)$, $x,y,z \in P$. \\ %i.e. it is a naww problem.
Let $\pairs((x,y,z)) = \{(x,y),(x,z),(y,z),(y,x),(z,x),(z,y)\}$\\
Let $A = \bigcup\limits_{r\in R} \pairs(r)$\\
Let $G = (V,E)$ be a directed graph, where $V \subseteq P$. We interpret an edge $(a,b) \in E$ to represent the statement `$a$ appears to the left of $b$ in a (one-dimensional) solution to the Ninja Assassin Wonderwall problem on $R$.' \\

\begin{codebox}
\Procname{$\proc{FindOrdering}(R)$}
\li Let $(a,b) \in A$. 
\li $G \gets (\emptyset, \emptyset)$
\li $ReverseChecked \gets \const{True}$
\li $G \gets \proc{MakeAssumption}(G, (a,b), R, ReverseChecked)$
\li \If $G$ is acyclic:
\li \Do
        \kw{output} $\proc{TopologicalSort}(G)$
\li \Else
        \kw{output} ``No ordering exists"
    \End
\end{codebox}

\begin{codebox}
\Procname{$\proc{MakeAssumption}(G, pair, R, ReverseChecked)$}
\li $W \gets G$ \Comment W is a working copy of G
\li $Q \gets \emptyset$
\li $\proc{Enqueue}(Q,pair)$
\li \While $\attrib{Q}{length} > 0$ 
\li \Do
        $(a,b) \gets \proc{Dequeue}(Q)$
\li     $\attrib{W}{E} \gets \attrib{W}{E} \cup \{(a,b)\}$
\li     Add to $W$ and $Q$ every edge (not already in $W$) that is implied by the edge $(a,b)$
\zi  \Comment e.g. If $(a,b) \in \attrib{W}{E}$ and $(a,b,c) \in R$, add edge $(b,c)$
\li     Add to $W$ and $Q$ the edges (not already in $W$) implied by the Transitive property.
\zi  \Comment e.g. If $(a,b),(b,c) \in \attrib{W}{E}$, add edge $(a,c)$
    \End % end While Q.length > 0
\li \If $W$ is acyclic
\li \Do
        \If $\attrib{W}{V} \isequal P$
\li     \Do
            \Return $W$ \Comment Success!
\li        \Else
\li         Let $(a,b) \in A \setminus \attrib{W}{E}$. \Comment Make another guess.
\li         $ReverseChecked \gets \const{false}$
\li         \Return $\proc{MakeAssumption}(W,(a,b), R, ReverseChecked)$
        \End
\li \ElseIf not $ReverseChecked$
\li \Do
        Let $(a,b) = pair$.
\li     $ReverseCheceked = \const{true}$
\li     \Return $\proc{MakeAssumption}(G, (b,a), R, ReverseChecked)$
\li \Else
        \Return $W$ \Comment both (a,b) and (b,a) lead to a contradiction. No solution.
    \End
\end{codebox} 

\begin{lem}
With $G$ and $R$ as described above, and $G$ acyclic, $\proc{MakeAssumption}$ returns a directed acyclic graph containing edge $(a,b)$ whose edges are consistent with both 
\begin{itemize}
    \item The transitive property: if $(a,b)$ and $(b,c)$ are edges, then $(a,c)$ is an edge.
    \item The constraints imposed by the Ninja Assassin Wonderwall problem: if $(a,b,c) \in R$, and $(a,b)$ is an edge, then $(b,c)$ is an edge.
\end{itemize}
as long as one exists. If no such DAG exists, it returns a cyclic graph.
\end{lem}
\begin{proof}
Suppose for some call of $\proc{MakeAssumption}(G,(a,b),R,ReverseChecked)$, the lemma is true for all subsequent recursive calls.
The \kw{while} loop on line 4 adds at least one edge to $W$, our working graph, so this call makes progress towards a solution. It also adds every edge implied by either the transitive property or the problem constraints.

If $W$ does not have a cycle and includes every player as a vertex, then we have created a dag that meets the requirements of the lemma, and we're done.

If $W$ does not have a cycle, but does not include every player, then we can make another guess. By assumption, this call maintains the truth of the lemma.

If $W$ has a cycle, the guess that $(a,b)$ was an edge ultimately created it. Therefore, $a$ is not to the left of $b$ in the ordering of players. So either $b$ is to the left of $a$ or there is no possible ordering. If we haven't checked $(b,a)$, the recursive call on line 19 will return the correct result. Otherwise, we know that there is no possible ordering and we return the cyclic graph $W$.
\end{proof}

\begin{thm}
Let $R$ be as described. $\proc{FindOrdering}(R)$ outputs an ordering of the $P$ players which satisfies the constraints of the Ninja Assassing Wonderwall problem represented by $R$, or outputs ``No ordering exists"
\end{thm}
\begin{proof}
Notice that for any ordering of the players which satisfies the constraints, its reverse also satisfies the constraints. Then we can assume for some $(a,b) \in A$ that $(a,b)$ is an edge in $G$. By the lemma, the call to $\proc{MakeAssumption}$ returns a DAG whose edges are consistent with the constraints of the Ninja Assassin Wonderwall problem if one exists, and a cyclic graph otherwise. So if $G$ is acyclic on line 5, a topological sort will produce an ordering consistent with the edges in $G$ and therefore the constraints of the Ninja Assassin Wonderwall problem. Otherwise, it outputs ``No ordering exists''.
\end{proof}
\end{comment}
\end{document}
