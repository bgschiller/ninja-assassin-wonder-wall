\documentclass{article}

\usepackage{amsmath, amssymb}
\usepackage[parfill]{parskip}
\usepackage[margin=1.25in]{geometry}
\pagestyle{empty}
\begin{document}
\begin{center}
\begin{Large}
The Ninja Assassin Wonderwall Project
\end{Large}\\
Senior Project proposal by Brian Schiler\\
Advisor: Tilman Glimm\\
\date{}
\vspace{20pt}
\end{center}
My senior project for the purposes of Graduation with Distinction from the Math Department concerns the mathematical investigation of a game called Ninja Assassin Wonderwall. The game, which is sometimes played at summer camps, is played with $n$ players. Each player chooses two other players  (without indicating their choices), one to be their ninja assassin and the other to be their wonderwall. When the game starts every player tries to keep their wonderwall between themself and their ninja assassin. 

\textbf{Definitions:} Let $P$ be the set of players, and let 
\[w: P \to P,\quad w(p) = \text{$p$'s wonderwall}\] 
\[n: P \to P,\quad n(p) = \text{$p$'s ninja assassin}\] 
be two functions. We also require $w(p) \neq p$, $n(p)\neq p$ and $w(p)\neq n(p)$. We define a solution $\phi$ as a function $\phi: P \to \mathbb{R}^n$ such that for every $p \in P$, $\phi(w(p))$ lies on the line segment between $\phi(p)$ and $\phi(n(p))$. (Thought of in terms of the game, $p$'s wonderwall is standing between $p$ and $p$'s ninja assassin.) 

For my senior project, I will:

\begin{itemize}
    \item Write and prove correct an algorithm for finding solutions to NAWW problems by looking for cycles in a directed graph.
    \item Devise a way to transform a NAWW instance into an instance of a well-known problem, such as a linear program or a CNF SAT problem.
    \item Prove that an instance of $p$ players that has a solution in $\mathbb{R}^n$ has a solution in $\mathbb{R}$. This will show that it is sufficient to search for solutions in one dimension.
    \item Explore whether the problem is np-complete, and attempt to prove one way or another.
    \item Use computer models to approximate the probability that an instance of a given size has a solution. Explore and conjecture about what happens to that probability for large numbers of players.
    \item Make animations to illustrate the game (and algorithm).
    \item Read about Steiner Triple Systems, implement a construction algorithm for STS(n), and use this to show that the algorithm has exponential worst-case time.
\end{itemize}

\begin{thebibliography}{9}

\bibitem{stinson}
    Douglas R. Stinson
    \emph{Combinatorial Designs: Constructions and Analysis}.
    Springer, New York,
    2004 edition.

\bibitem{cameron}
    Peter J. Cameron
    \emph{Combinatorics: Topics, Techniques, Algorithms}.
    Cambridge University Press,
    1995 edition.

\bibitem{clrs}
    T.~H. Cormen, C.~E. Leiserson, R.~L. Rivest, and C.~Stein
    \emph{Introduction to Algorithms}.
    The MIT Press,
    2009 edition.


\end{thebibliography}

\end{document}
